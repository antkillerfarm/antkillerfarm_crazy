% arara: pdflatex
% arara: animate: {density: 240, delay: 60, loop: 0, options: ['-alpha', 'remove']}
\documentclass[tikz, border=3.14mm]{standalone}
\usetikzlibrary{arrows.meta,decorations.markings}
\usepackage{ifthen}
\usetikzlibrary{math}

\begin{document}

\pgfdeclarelayer{background}
\pgfsetlayers{background,main}

\foreach \Y in {1,...,3}
    \foreach \X in {1,...,3}
{
\begin{tikzpicture}[draw=black!50]
\tikzstyle{neuron}=[rectangle, draw=black, fill=black!25, minimum size=10pt]
\tikzstyle{input neuron}=[neuron, fill=green!50];
\tikzstyle{output neuron}=[neuron, fill=blue];
\tikzstyle{kernel neuron}=[neuron, fill=red!50];
\tikzstyle{kline} = [black, thick]

% Draw the input layer nodes
\tikzmath{
    \scalenn = 0.4;
    \inputsize = 10;
    \outputsize = 8;
    \kernelsize = 3;
    \xstart = \X - 1;
    \ystart = \Y - 1;
    \xend = \xstart + \outputsize + 1;
    \yend = \ystart + \outputsize + 1;
    int \bxstart, \bystart, \bxend, \byend;
    \bxstart = \xstart + 1;
    \bystart = \ystart + 1;
    \bxend = \xend - 1;
    \byend = \yend - 1;
}

\foreach \x in {1,...,\inputsize}
    \foreach \y in {1,...,\inputsize}
        \ifthenelse {\x > \xstart \and \x < \xend
            \and \y > \ystart \and \y < \yend}
        {
            \node[output neuron] (I-\x-\y) at (\x * \scalenn, -\y * \scalenn) {};
        }
        {% else
            \node[input neuron] (I-\x-\y) at (\x * \scalenn, -\y * \scalenn) {};
        };

\foreach \x in {1,...,\kernelsize}
    \foreach \y in {1,...,\kernelsize}
        \node[kernel neuron, xshift=7cm, yshift=-1cm] (K-\x-\y) at (\x * \scalenn, -\y * \scalenn) {};

\node [scale=0.5, xshift=13.5cm, yshift=-7cm](LegendTable) {
    \begin{tabular}{l|l} 
    \textbf{Name} & \textbf{[Height, Width]} \\
    \hline
    Input & [10, 10] \\
    Output(gradient) & [8, 8] \\
    Kernel(gradient) & [3, 3] \\
    \end{tabular}
};

\draw [kline] (I-\bxstart-\bystart.north west) -- (K-\X-\Y.north west);
\draw [kline] (I-\bxend-\bystart.north east) -- (K-\X-\Y.north east);
\draw [kline] (I-\bxstart-\byend.south west) -- (K-\X-\Y.south west);
\draw [kline] (I-\bxend-\byend.south east) -- (K-\X-\Y.south east);

\begin{pgfonlayer}{background}
    % Compute a few helper coordinates
    \path (I-1-1.west |- I-1-1.north)+(-0.3,0.3) node (a) {};
    \path (I-1-1.west |- I-1-1.north)+(+8.5,-4.3) node (b) {};
    \path[fill=yellow!20,rounded corners, draw=black!50, dashed]
        (a) rectangle (b);
\end{pgfonlayer}

\end{tikzpicture}
}
\end{document}
